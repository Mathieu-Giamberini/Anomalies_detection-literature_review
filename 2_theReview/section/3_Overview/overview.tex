\documentclass[../../main/main.tex]{subfiles}

\begin{document}
    \begin{figure*}[hp]
        \makebox[\textwidth]{\subfile{./1_tree/tree.tex}}
        \caption{Anomaly detection method categorization}
    \end{figure*}

    Anomaly detection method can be categorize in many way. In this paper its divided in two parts,
    general method and useful ticks to improve theses. The former represent general ways to 
    solve the AD problem. Its compose of tree categorizes : forecasting, clustering and index monitoring.
    The latter are so called add-on. The papers in this category present a AD method with some tricks which,
    in our mind can be applied for any method to improve it.
    \subsection{Forecasting Methods}
        When working all day long with some equipment, you know it so well that if asked you could almost 
        predict the sound it will make in some close future. But if suddenly sound doesn't match 
        your expectation  you know there is some things wrong.
        That is the core idea for the forecasting method, predict the near future using some generative method
        and if the error go over some threshold the data point is labeled as an anomaly. Example of those technic 
        may be LSTM [ref claim LSTM is the best] or using time convolution (TCN) [Ref TCN].
    
    \subsection{Clustering method}
        While storing metric screw in the ware house some one pick an imperial one, he want find any box to put it in.
        Therefor he will conclude a error in the shipment, even if he didn't know this type of screw. This is the core 
        idea for the clustering method, define clusters, if a new data point can't fit in one of them it labeled as an anomaly.  
        There is multiple way to define the cluster. A good part use some define distance [ref DTW] function and use a variant of K-nearest-neighbor (KNN)
        [ref DBSCAN]. In a similar fashion use the density of the data point in relation with the training data which will 
        give less strick categorizes [ref LOF]. An other method is to estimate the probability density function (PDF) 
        and put a threshold on low values. [ref DAGMM]
        
    %\subfile{./plan.tex}
    
\end{document}