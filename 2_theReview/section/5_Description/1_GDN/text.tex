\documentclass[../description.tex]{subfiles}

\begin{document}
    When using "distance to prediction" for AD in a multivariate settings, it's 
    sensible to assume that related sensors have redundant information 
    (e.g. a pressure and a extension sensors for a hydraulic cylinder).
    So if there measurement was put together to predict the next ones 
    it could improve their prediction and so reduce the detection threshold. This is a base idea of the GDN architecture in ~\cite{Deng2021gdn}.
    Mark the related sensor in a directed Graph
    and apply an attention layer on the past measurement 
    of the neighborhoods to predict the next ones.
    Here is some short assertion and nice propriety of this approach :
    \begin{itemize}
        \item Even if every sensor of the TBM are in the graph, the AD is still localize to each sensor.
        \item The training data is assume to be free of anomaly.
        \item Prior knowledge of relation between sensor can be embedded in the graph by restricted some relationship. 
        This can lower the complexity and the time of training.
    \end{itemize}

    To test their method the authors used dataset of simulated attack on water treatment physical test-bed systems, 
    the Secure Water Treatment (SWaT) and Water Distribution (WADI) dataset. As a base line they use other method which we discus in other Section.
    
    \begin{table}
        \centering\fontsize{9}{11}\selectfont
        % \resizebox{\textwidth}{!}{
        \begin{tabular}{rrrrrrr}
            \toprule
              & \multicolumn{3}{c}{\textbf{\texttt{SWaT}}} & \multicolumn{3}{c}{\textbf{\texttt{WADI}}} \\
             \cmidrule(r){2-4} \cmidrule(r){5-7}
             \textbf{Method} & $\mathbf{\mathsf{Prec}}$ & $\mathbf{\mathsf{Rec}}$ & $\mathbf{\mathsf{F1}}$ & $\mathbf{\mathsf{Prec}}$ & $\mathbf{\mathsf{Rec}}$ & $\mathbf{\mathsf{F1}}$ \\
             \midrule
             \textsc{\ac{PCA}}                & 24.92 & 21.63 & 0.23 & 39.53 & 5.63  & 0.10 \\
             \textsc{\ac{KNN}}                & 7.83  & 7.83  & 0.08 & 7.76  & 7.75  & 0.08 \\
             \textsc{FB}                      & 10.17 & 10.17 & 0.10 & 8.60  & 8.60  & 0.09 \\
             \textsc{\ac{AE}}                 & 72.63 & 52.63 & 0.61 & 34.35 & 34.35 & 0.34 \\
             \textsc{\footnotesize{DAGMM}}    & 27.46 & 69.52 & 0.39 & 54.44 & 26.99 & 0.36 \\
             \textsc{\footnotesize{LSTM-VAE}} & 96.24 & 59.91 & 0.74 & 87.79 & 14.45 & 0.25 \\
             \textsc{\footnotesize{MAD-GAN}}  & 98.97 & 63.74 & 0.77 & 41.44 & 33.92 & 0.37 \\
             \midrule
             \textsc{\ac{GDN}} & \textbf{99.35} & \textbf{68.12} & \textbf{0.81} &\textbf{97.50} & \textbf{40.19} & \textbf{0.57} \\
             \bottomrule
        \end{tabular}%}
        \caption{Anomaly detection accuracy in terms of precision(\%), recall(\%), and F1-score, on two datasets with ground-truth labelled anomalies. Part of the results are from ~\cite{li2019mad}.}
        \label{tab:performance}
    \end{table}
    
\end{document}