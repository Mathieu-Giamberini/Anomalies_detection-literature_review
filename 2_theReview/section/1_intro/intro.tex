\documentclass[../../main/main.tex]{subfiles}

\begin{document}
    %\subfile{plan.tex}
    %\subfile{definition/definition-AD.tex}
    %to edit, if no time use binary class definition 

    By default TBM (Tunnel Boring Machines) are complicated technical objects and involved 
    a lot of technologies working together. Any defect in the chain could stop the entire machine. There for detecting those 
    before they become a real issue can save a lot of time and resources. Fortunately, the number of sensor on the machines keep raising 
    and thanks to platform like HK-connect their data can be access any where by any authorizes users. These data must carry a lot of information 
    regarding the sate of the machine, which beg the question on how to extract it to detect anomaly. At time of writing the task is mostly done
    by experience users of the TBM, e.g. a weird noise in the hydraulic pack, an odd torque increase on one of the
    main drive motors, and so on. This method is clearly limited and use only a small portion of the data stream available, thanks fully the 
    problem of anomaly detection is quite old and with the raise of machine leaning technic, keep getting better. \\
    Therefor in this paper we will, have a first look of at the littery landscape of Anomaly detection method (ADM).
    First lets define exactly what we mean by that, given a multivariate time series: 
    \[\mathcal{X} = (\textbf{x}_1, \dots \textbf{x}_t, \dots \textbf{x}_T) \in \mathbb{R}^{N \times T}\]
    with $N$ the number of sensors and $T$ the number of temporal indices. The method $\mathcal{M} $ will return a binary category for each data point 
    telling where and when there is anomalies, ie:
    \[\mathcal{M} : \mathbb{R}^{N \times T} \to \left\{0, 1\right\} ^{N \times T}\]
    We need to address some looses ends with this definition. First, in most cases, their is no already labeled data to train the method on
    and labeling it by hand is not feasible. So the method must be unsupervised, which raise the need for a systemic definition of anomaly. 
    Most papers (ref) consider the training data to be normal and define some anomaly score which tel how far is the new data. Here the methods will 
    be categorize base on the way this score is define.\\
    %TODO ref
    In our context, the training data is previous tunnel bore, which certainly contains anomaly in it. Therefor we will also discus some 
    papers which tyred to address this issue by modifying the methods. 
\end{document}